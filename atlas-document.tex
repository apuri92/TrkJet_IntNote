%-------------------------------------------------------------------------------
% This file provides a skeleton ATLAS document.
%-------------------------------------------------------------------------------
% Specify where ATLAS LaTeX style files can be found.
\newcommand*{\ATLASLATEXPATH}{latex/}
% Use this variant if the files are in a central location, e.g. $HOME/texmf.
% \newcommand*{\ATLASLATEXPATH}{}
%-------------------------------------------------------------------------------
\documentclass[UKenglish,texlive=2013]{\ATLASLATEXPATH atlasdoc}
% The following command is needed by arXiv to ensure use of pdflatex.
% It should be included in the first 5 lines of the preamble.
% \pdfoutput=1
% The language of the document must be set: usually UKenglish or USenglish.
% british and american also work!
% Commonly used options:
%  texlive=YYYY          Specify TeX Live version (2013 is default).
%  atlasstyle=true|false Use ATLAS style for document (default).
%  coverpage             Create ATLAS draft cover page for collaboration circulation.
%                        See atlas-draft-cover.tex for a list of variables that should be defined.
%  cernpreprint          Create front page for a CERN preprint.
%                        See atlas-preprint-cover.tex for a list of variables that should be defined.
%  PAPER                 The document is an ATLAS paper (draft).
%  CONF                  The document is a CONF note (draft).
%  PUB                   The document is a PUB note (draft).
%  txfonts=true|false    Use txfonts rather than the default newtx - needed for arXiv submission.
%  paper=a4|letter       Set paper size to A4 (default) or letter.

%-------------------------------------------------------------------------------
% Extra packages:
\usepackage{\ATLASLATEXPATH atlaspackage}
% Commonly used options:
%  biblatex=true|false   Use biblatex (default) or bibtex for the bibliography.
%  backend=biber         Use the biber backend rather than bibtex.
%  subfigure|subfig|subcaption  to use one of these packages for figures in figures.
%  minimal               Minimal set of packages.
%  default               Standard set of packages.
%  full                  Full set of packages.
%-------------------------------------------------------------------------------
% Style file with biblatex options for ATLAS documents.
\usepackage{\ATLASLATEXPATH atlasbiblatex}

% Package for creating list of authors and contributors to the analysis.
\usepackage{\ATLASLATEXPATH atlascontribute}

% Useful macros
\usepackage{\ATLASLATEXPATH atlasphysics}
% See doc/atlas-physics.pdf for a list of the defined symbols.
% Default options are:
%   true:  journal, misc, particle, unit, xref
%   false: BSM, hion, math, process, other, texmf
% See the package for details on the options.

% Files with references for use with biblatex.
% Note that biber gives an error if it finds empty bib files.
\addbibresource{atlas-document.bib}
\addbibresource{bibtex/bib/ATLAS.bib}

% Paths for figures - do not forget the / at the end of the directory name.
\graphicspath{{logos/}{figures/}}

% Add you own definitions here (file atlas-document-defs.sty).
\usepackage{atlas-document-defs}

%-------------------------------------------------------------------------------
% Generic document information
%-------------------------------------------------------------------------------

% Title, abstract and document 
\input{atlas-document-metadata}
% Author and title for the PDF file
\hypersetup{pdftitle={ATLAS draft},pdfauthor={The ATLAS Collaboration}}

%-------------------------------------------------------------------------------
% Content
%-------------------------------------------------------------------------------
\begin{document}

\maketitle

\tableofcontents

% List of contributors - print here or after the Bibliography.
%\PrintAtlasContribute{0.30}
%\clearpage

%-------------------------------------------------------------------------------
\section{Introduction}
\label{sec:intro}
%-------------------------------------------------------------------------------

Place your introduction here.


%-------------------------------------------------------------------------------
\section{ATLAS detector}
\label{sec:detector}
%-------------------------------------------------------------------------------

The ATLAS detector~\cite{PERF-2007-01} ...
% \input{atlas-detector}


%-------------------------------------------------------------------------------
\section{Results}
\label{sec:result}
%-------------------------------------------------------------------------------

Place your results here.

% All figures and tables should appear before the summary and conclusion.
% The package placeins provides the macro \FloatBarrier to achieve this.
% \FloatBarrier


%-------------------------------------------------------------------------------
\section{Conclusion}
\label{sec:conclusion}
%-------------------------------------------------------------------------------

Place your conclusion here.


%-------------------------------------------------------------------------------
\section*{Acknowledgments}
%-------------------------------------------------------------------------------

%\input{acknowledgements/Acknowledgements}

The \texttt{atlaslatex} package contains the acknowledgments that were valid 
at the time of the release you are using.
These can be found in the \texttt{acknowledgments} subdirectory.
When your ATLAS paper or PUB/CONF note is ready to be published,
download the latest set of acknowledgments from:\\
\url{https://twiki.cern.ch/twiki/bin/view/AtlasProtected/PubComAcknowledgements}

The supporting notes for the analysis should also contain a list of contributors.
This information should usually be included in \texttt{mydocument-metadata.tex}.
The list should be printed either here or before the table of contents.


%-------------------------------------------------------------------------------
\clearpage
\appendix
\part*{Appendix}
\addcontentsline{toc}{part}{Appendix}
%-------------------------------------------------------------------------------

In a paper, an appendix is used for technical details that would otherwise disturb the flow of the paper.
Such an appendix should be printed before the Bibliography.


%-------------------------------------------------------------------------------
% If you use biblatex and either biber or bibtex to process the bibliography
% just say \printbibliography here
\printbibliography
% If you want to use the traditional BibTeX you need to use the syntax below.
%\bibliographystyle{bibtex/bst/atlasBibStyleWoTitle}
%\bibliography{atlas-document,bibtex/bib/ATLAS}
%-------------------------------------------------------------------------------

%-------------------------------------------------------------------------------
% Print the list of contributors to the analysis
% The argument gives the fraction of the text width used for the names
%-------------------------------------------------------------------------------
\clearpage
\PrintAtlasContribute{0.30}

%-------------------------------------------------------------------------------
\clearpage
\appendix
\part*{Auxiliary material}
\addcontentsline{toc}{part}{Auxiliary material}
%-------------------------------------------------------------------------------

In an ATLAS paper, auxiliary plots and tables that are supposed to be made public 
should be collected in an appendix that has the title \enquote{Auxiliary material}.
This appendix should be printed after the Bibliography.
At the end of the paper approval procedure, this information can be split into a separate document
-- see \texttt{atlas-auxmat.tex}.

In an ATLAS note, use the appendices to include all the technical details of your work
that are relevant for the ATLAS Collaboration only (e.g.\ dataset details, software release used).
This information should be printed after the Bibliography.

\end{document}
