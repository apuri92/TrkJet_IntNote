% !TEX root = trackjet_intnote.tex

Heavy ion collisions at collider energies are performed in order to produce
and study QCD matter at high temperature, the quark-gluon plasma (QGP).  
Measurements of jets in such 
collisions are powerful tools to determine the properties of this matter
by measuring the modification of jet production and fragmentation 
after the jets have traversed the hot QCD matter.
The rates of jet production~\cite{Abelev:2013kqa,Aad:2014bxa,Khachatryan:2016odn} and their correlations~\cite{Aad:2010bu}
are observed to be modified in a centrality dependent manner in \PbPb\ collisions at
2.76~TeV and 5.02~TeV\cite{ATLAS:2017wvp}.
In addition, the longitudinal momentum
 distribution of charged particles within jets measured is observed to be modified as 
well~\cite{Aad:2014wha,Chatrchyan:2014ava, Aaboud:2017bzv,PbPb5TeVIntNote}.
Significant modifications of the jet fragmentation are observed, including an
excess of particles with transverse momentum (\pT) less than about 4~GeV.  The excess
is observed to increase with both centrality and the transverse momentum of the jet (\pTjet).
However, these measurements are insensitive to the angular distribution of charged
particles within the jet and particles outside of the jet cone.  This information
is crucial to understanding how the jet interacts with the hot QCD matter
created in \pbpb\ collisions at the LHC. The measurement of the particle distributions at large angles outside the jet cone provides information about the flow of transverse momentum lost by the jet due to the jet quenching phenomena. The angular distribution of the soft particles around the jet is
a key feature of models of the interaction of the jets with the
hot QCD matter~\cite{Blaizot:2014ula,Brewer:2017fqy} and the
response of the QCD matter to the propagation of the jet~\cite{Tachibana:2017syd,Yan:2017rku}.

In this note, a measurement of the angular and transverse momentum distributions of tracks around the jet
axis is presented.  The distributions are presented as a function of centrality
and \pTjet.  In order to quantify the effects due to the presence of the hot
QCD matter, the same quantities are measured in \pp\ collisions at the same collision 
energy to provide a baseline of unmodified jets.

In this analysis, we extend previous ATLAS fragmentation studies by measuring of the angular distributions of
charged particles around the jet axis, including those outside the jet cone. While we aim to eventually study distances up to $\rvar < 1.2$, this preliminary internal note only explores distances up to $\rvar < 0.6$ because of the large underlying event and associated uncertainties.
Previous measurements from CMS~\cite{Khachatryan:2016tfj,CMSPASHIN16020}
have studied similar correlations.  This measurement extends those
by including the \pTjet\ dependence of the correlations and
by utilizing unfolding such that detector effects are removed from the final results.


The content of this note is as follows: 
Section~\ref{sec:trkjet_corr_measurement} defines basic quantities entering the 
measurement. 
Section~\ref{sec:used_data} 
summarizes the data that have been used for this study. 
Section~\ref{sec:event_selection} discusses the event selection and 
centrality selection. 
Section~\ref{sec:reconstruction} describes the methods used for 
the jet reconstruction in heavy ion collisions. 
Section~\ref{sec:cuts_corrections} describes the cuts 
and corrections used for jets and charged particles in the analysis. 
Section~\ref{Sec:systematic} describes the uncertainties.
Section~\ref{sec:results} describes the results of the measurement, and 
Section~\ref{sec:summary} provides a summary of the analysis.


